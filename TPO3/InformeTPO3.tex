\documentclass{article}
\usepackage[utf8]{inputenc}
\usepackage[spanish]{babel}
\usepackage[graphicx]
\title{\Huge\textbf{Informe TPO N°3}\\ Lenguajes declarativos}

\author{Apablaza Fabio}

\date{26 junio del 2020}

\begin{document}
\maketitle
\newpage


%Pagina 1 la introduccion al informe
\section{Introduccion}
El siguiente informe explica la implementacion y resolucion de un problema de combinacion de ajedrez.

\newpage

\section{Descripcion del problema}
\begin{Center}
Tomando como base el trabajo realizado en el TPO N°2 sobre programacion Logica con Restricciones escribir un programa que genere una interfaz web visual utilizando las librerias vistas en las clases de teoria.\\
Se espera que el programa permita cargar un tablero - una configuracion de casillas con una cantidad de ataques definida en cada una - desde un archivo, encontrar y representar la solucion para el mismo.\\
A modo guia se sugiere revisar el ejemplo del juego Sudoku que se encuentra en la plataforma.\\
Ademas de la configuracion original del tablero dad en el enunciado del TPO N°2 pueden utilizar el siguiente tablero para probar:\\
%Imagen del TPO

\end{Center}

\newpage
%Descripcion de la solucion
\section{Analisis Breve de la solucion planteada}

\newpage
\section{Conclusiones}

\newpage
\section{Bibliografia}

\end{document}
